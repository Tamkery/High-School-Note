\documentclass{book}
%\usepackage[utf8]{inputenc}
\usepackage[UTF8]{ctex}
\usepackage{chemfig}
\usepackage{ulem}%删除线
\usepackage{amsmath} %分段函数
\usepackage{amsmath} %\text{ } math
%化学
\usepackage{tikz}
\usepackage{chemfig}
\usepackage[version=3,arrows=pgf]{mhchem}
\usepackage{verbatim} % 化学


\title{寻松点点的高中化学笔记}
\author{Tamkery }

\begin{document}
\maketitle
\tableofcontents
%\begin{abstract}
%\end{abstract}



\newcommand\ncoord[2][0,0]{%
    \tikz[remember picture,overlay]{\path (#1) coordinate (#2);}%
}

\newcommand\ccoord[1]{\ncoord[0.5em,0.9em]{#1}}%
\newcommand\bcoord[1]{\ncoord[0.4em,-0.3em]{#1}}%
\newcommand\dcoord[1]{\ncoord[0.9em,1.5em]{#1}}%
\newcommand\ecoord[1]{\ncoord[0.9em,-0.3em]{#1}}%

\tikzset{
    oxidation/.style={thick,red!80!black},
    reduction/.style={thick,blue!80!black},
}



~



\part{必修一}

\chapter{第一章\ 从实验学化学}


\section{第一节\  化学实验基本方法}

\section{第二节\  化学计量}	


\chapter{第二章\ 化学物质及其变化}

\section{第一节\  物质的分类}

\section{第二节\  离子反应}

\section{第三节\  氧化还原反应}

\subsection{四、氧化还原反应中电子转移的表示方法}
方法一、双线桥法: \\
\[
\ce{{\ccoord{1}} Na + {\bcoord{c}} Cl -> {\ccoord{2}} Na+  + {\bcoord{d}} Cl-}
\]

\tikz[overlay,remember picture] {
    \draw[oxidation,->] (1) -- ++(0,0.7em) -| (2)
        node[above, near start] {氧化反应};
    \draw[reduction,->] (c) -- ++(0,-0.7em) -| (d)
        node[below, near start] {还原反应};
}
\\[0.5cm] 







\part{选修五\ 有机化学基础}

\chapter{第一章\ 认识有机物}

\section{第一节\ 有机化合物的分类}	

\subsection{一、有机物的组成}
1. 按元素组成 \par
有机物:烃和烃的衍生物 \par
(烃:烷、烯、炔、芳香烃)
\newline \par
2. 按碳骨架分\par
链状化合物:如:\chemfig{CH_3CH_2CH_2CH_3}
\newline \par
环状化合物:\par
脂环化合物 如:\chemfig{*6(------)}\hspace{0.8mm} \par
芳香化合物 如:\chemfig{*6(-=-=-=) }
\newline \par

3. 按官能团分

\begin{table}
% table caption is above the table
\caption{官能团}
\label{tab:1}       % Give a unique label

% For LaTeX tables use
\begin{tabular}{lll}

\hline\noalign{\smallskip}

 类别 & 官能团 & 例子  \\

\noalign{\smallskip}\hline\noalign{\smallskip}

 烷烃 & (没有官能团) & 甲烷:\chemfig{CH_4} \\

 烯烃 & \chemfig{C(-[3])(-[5])=C(-[1])(-[7])} & 乙烯: \chemfig{CH_2=CH_2} \\

 炔烃 & \chemfig{C(-[4])~C(-[0])} & 乙炔: \chemfig{CH~CH} \\

 芳香烃 & (没有官能团) & 苯 \sout{(易错写为笨)}:\chemfig{*6(-=-=-=)} \\

 卤代物 & \chemfig{-X} \ (X表示卤素原子)  &溴乙烷:\chemfig{CH_3CH_2Br} \\

 醇 & \chemfig{-OH} \ 羟基 & 乙醇:\chemfig{CH_3CH_2oh} \\

 
 酚 & \chemfig{-OH} \ 羟基 &苯酚: \chemfig{*6(-=-(-[1]OH)=-=)}  \\

 醚 &\chemfig{C(-[3])(-[4])(-[5])-O-C(-[1])(-[0])(-[7])} \ 醚键& 乙醚: \chemfig{CH_3CH_2OCH_2CH_3 }   \\ 
    
 醛 & \chemfig{C(-[0]H)(=[2]O)(-[4]) } \ 醛基& 甲醛:\chemfig{CH_3 \hspace{0.5cm}-C(-[0]H)(=[2]O)(-[4]) }\\ 

 酮 & \chemfig{C(-[0])(=[2]O)(-[4])}\hspace{0.5cm} 羰(tang)基 & 丙酮:\chemfig{CH_3-C(=[2]O)-CH_3}\\

 羧基 &\chemfig{ C(-[0]OH)(=[2]O)(-[4])} \hspace{0.5cm} 羧(suo)基 
& 乙酸(98\%冰醋酸) \hspace{0.5cm}
\chemfig{CH_3\hspace{0.5cm}-C(-[0]OH)(=[2]O)(-[4])}\\

 脂 & \chemfig{ C(=[2]O)(-[4])-O-R} \hspace{0.5cm} 脂基    
&乙酸乙酯:\chemfig{CH_3-C(=[2]O)-O-C_2H_5}   \\

% & & \\
\noalign{\smallskip}\hline
	\end{tabular}
\end{table}

\newpage
\section{第二节 \ 有机化学的结构特点}
\subsection{一、碳原子的成键特点}  

1. 一个碳原子只能形成四个共价键 \par
2. 可以与其他非金属原子形成共价键 \par
3. 碳原子键可以形成单键、双键、三键 \par
4. 可成键、可成环 \par
(有机物种类繁多的原因之一)

注:H:一个共价键 \ N:三个共价键 \ O:两个共价键
\newline \par
\subsection{二、有机物的同分异构体现象—有机物种类繁多的原因}
%\begin{equation}
$$\text{同分异构类型:}
	\begin{cases}  \text{1. 碳键异构:碳键骨架不同} \\
						\text{2. 位置异构:官能团位置不同} \\
 						\text{3. 官能团异构:分子式相同,但官能团不同}
	\end{cases}$$
%\end{equation}
\newline  \par
碳键异构:\chemfig{C_4H_{10}}\\
\chemfig{CH_3-CH_2-CH_2-CH_3 } \quad \chemfig{CH_3-CH(-[2]CH_3)-CH_3 } 
\\
\newline \par
官能团异构:\chemfig{C_2H_6O}\par
\chemfig{C(-[2]H)(-[4]H)(-[6]H)-C(-[2]H)(-[6]H)-OH } \quad   \chemfig{CH_3-O-CH_3}\\
\newline \par
位置异构:\chemfig{C_4H_8} \\
\chemfig{CH_2=CH_2-CH_2-CH_3} \quad \chemfig{CH_3-CH_2=CH_2-CH_3} \\

注:\\
1. 同数的醇和醚可互为同分异构体\\
2. 相同碳原子的酸和脂互为同分异构体\\
3. 同数碳原子的氨基酸和硝基烷互为同分异构体\\


\section{第三节\ 有机物的命名}

\subsection{一、烷烃命名}
烷基:\chemfig{C_nH_{2n+1}} \\
\newline 
甲基:\chemfig{-CH_3} \\
乙基:\chemfig{-CH_2-CH_3} \\
丙基: \chemfig{-CH_2-CH_2-CH_3} \\
异丙基: \chemfig{CH_3-CH(-[6])-CH_3 }


\subsection{二、命名步骤}
烷烃类有机物的命名步骤和原则:\\
1. 选主链,称“某烷” ——“最长原则”  \\
2. 编号位,定支链 ——“最近原则” \\
3. 取代基写在前,标位置,连短线 \\
4. 相同基合并写 ——“最小取代基位号之和最小” \\
5. 不同基简到繁 ——“最简:当主链有两个以上时,选择取代基最简单的”

\subsection{三、烯烃、炔烃命名}
\newpage
\subsection{四、苯的同系物命名}
苯环分子中的H被烷基取代 \\
\newline 
命名:以苯环为母体 \\
1. 有两个烷基:“邻(居);间(隔);对(称)” \\
2. 有三个烷基:“连偏均 ” \\



\chemname{\chemfig{*6(-(-[6]CH_3) =(-[7]CH_3) -(-[1]CH_3) =- = ) } }
{连三甲苯}  \quad
\chemname{\chemfig{*6(-(-[6]CH_3) = -(-[1]CH_3) =(-[2]CH_3) - = )} }
{偏三甲苯 } \ 
\chemname{\chemfig{*6((-[5]CH_3)- =(-[7]CH_3) - =(-[2]CH_3) - =  )} }
{均三甲苯} 







\end{document}
