\documentclass{book}
%\usepackage[utf8]{inputenc}
\usepackage[UTF8]{ctex}
\usepackage{chemfig}
\usepackage{ulem}%删除线

\title{寻松点点的高中化学笔记}
\author{Tamkery }

\begin{document}
\maketitle
\tableofcontents
%\begin{abstract}
%\end{abstract}


\part{选修五\ 有机化学基础}

\chapter{第一章\ 认识有机物}

\section{第一节\ 有机化合物的分类}	

\subsection{一、有机物的组成}
1. 按元素组成 \par
有机物:烃和烃的衍生物 \par
(烃:烷、烯、炔、芳香烃)
\newline \par
2. 按碳骨架分\par
链状化合物:如:\chemfig{CH_3CH_2CH_2CH_3}
\newline \par
环状化合物:\par
脂环化合物 如:\chemfig{*6(------)}\hspace{0.8mm} \par
芳香化合物 如:\chemfig{*6(-=-=-=) }
\newline \par

3. 按官能团分

\begin{table}
% table caption is above the table
\caption{官能团}
\label{tab:1}       % Give a unique label

% For LaTeX tables use
\begin{tabular}{lll}

\hline\noalign{\smallskip}

 类别 & 官能团 & 例子  \\

\noalign{\smallskip}\hline\noalign{\smallskip}

 烷烃 & (没有官能团) & 甲烷:\chemfig{CH_4} \\

 烯烃 & \chemfig{C(-[3])(-[5])=C(-[1])(-[7])} & 乙烯: \chemfig{CH_2=CH_2} \\

 炔烃 & \chemfig{C(-[4])~C(-[0])} & 乙炔: \chemfig{CH~CH} \\

 芳香烃 & (没有官能团) & 苯 \sout{(易错写为笨)}:\chemfig{*6(-=-=-=)} \\

 卤代物 & \chemfig{-X} \ (X表示卤素原子)  &溴乙烷:\chemfig{CH_3CH_2Br} \\

 醇 & \chemfig{-OH} \ 羟基 & 乙醇:\chemfig{CH_3CH_2oh} \\

 
 酚 & \chemfig{-OH} \ 羟基 &苯酚: \chemfig{*6(-=-(-[1]OH)=-=)}  \\

 醚 &\chemfig{C(-[3])(-[4])(-[5])-O-C(-[1])(-[0])(-[7])} \ 醚键& 乙醚: \chemfig{CH_3CH_2OCH_2CH_3 }   \\ 
    
 醛 & \chemfig{C(-[0]H)(=[2]O)(-[4]) } \ 醛基& 甲醛:\chemfig{CH_3 \hspace{0.5cm}-C(-[0]H)(=[2]O)(-[4]) }\\ 

 酮 & \chemfig{C(-[0])(=[2]O)(-[4])}\hspace{0.5cm} 羰(tang)基 & 丙酮:\chemfig{CH_3-C(=[2]O)-CH_3}\\

 羧基 &\chemfig{ C(-[0]OH)(=[2]O)(-[4])} \hspace{0.5cm} 羧(suo)基 
& 乙酸(98\%冰醋酸) \hspace{0.5cm}
\chemfig{CH_3\hspace{0.5cm}-C(-[0]OH)(=[2]O)(-[4])}\\

 脂 & \chemfig{ C(=[2]O)(-[4])-O-R} \hspace{0.5cm} 脂基    
&乙酸乙酯:\chemfig{CH_3-C(=[2]O)-O-C_2H_5}   \\

% & & \\
\noalign{\smallskip}\hline

\end{tabular}

\end{table}


\end{document}
